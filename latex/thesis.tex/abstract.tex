%Abstract Page 
%*******************************************************
% Abstract+Sommario
%*******************************************************

\pdfbookmark{Abstract}{Abstract}
\begingroup
\let\clearpage\relax
\let\cleardoublepage\relax
\let\cleardoublepage\relax

\chapter*{Abstract}
With every generation of new media, advertisers find new ways to persuade and manipulate consumers to buy. Online behavioral advertising is concerned with fine-grained targeting of consumers by examining and tracking their online behavior over time. The ecosystem of behavioral advertising contains both interactive ads as well as mechanisms designed to enhance and protect privacy. Despite efforts by the Federal Trade Commission (FTC) and Advertising Industry to make privacy controls more transparent and easy to use, there is much confusion about how these work. 

The aim of this thesis is to prove that certain problems and confusions that users face interacting with technology associated with online behavioral advertising can best be understood as problems in discourse understanding. Through the use of quantitative methods, I demonstrate that users are subject to the same context-dependent aspects of meaning that arise with the use of language. 

With the rise of new interactions fueled by behavioral advertising, I suggest that advertisers may look toward exploiting new opportunities for manipulating meaning and understanding. Because the same cognitive architecture that supports language understanding also supports and facilitates understanding of user interfaces, discourse theory helps illuminate effects that small changes in context may have on user understanding during the course of interaction.


\vfill
\endgroup			
\vfill
