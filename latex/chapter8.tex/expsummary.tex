%% LyX 2.0.6 created this file.  For more info, see http://www.lyx.org/.
%% Do not edit unless you really know what you are doing.
%\documentclass[english]{article}
%\usepackage[T1]{fontenc}
%\usepackage[latin9]{inputenc}

%\makeatletter

%%%%%%%%%%%%%%%%%%%%%%%%%%%%%% LyX specific LaTeX commands.
%% Because html converters don't know tabularnewline
%\providecommand{\tabularnewline}{\\}
%\@fpsep\textheight
%\makeatother
%\usepackage{babel}
%\begin{document}
\footnotesize
\begin{longtable}{p{2cm} p{9cm}}
%\hline
\caption{Summary Results} \\
\label{summary-results} \\
%\hline
%Study No. & Description \\ 
%\hline
1A Design & 2x2x2 between groups design where the control group is presented a set of textual expressions and asked to answer questions about their meaning. Treatment groups are presented with textual expressions or a dialog box expressing the same set of choices and asked the same questions. Confounding factors include both a privacy (knowledge) bias and also deontic force.
\tabularnewline
1A Results & Significant difference between text and graphics for implicature (p<.01). Participants are even more likely to interpret an implicature in the graphical conditions than in the textual condition. Also, a significant interaction in the condition where deontic force accompanies `'pictures' (p<.05). When there is no privacy bias (less knowledge), the conditional implicature has larger effect.
\tabularnewline
\hline
1B Design & Between groups design where two groups are presented
with a cookie banner and later asked about whether or not they believe the website placed \textquotedbl{}cookies\textquotedbl{} in their browser. The treatment group is presented with feedback about the consequence of their action (or non-action) following presentation of the banner. 
\tabularnewline
1B Results & Significant difference in understanding with presentation of feedback (p<.001). With no feedback, the odds of implicature are 6.6 times more likely.
\tabularnewline
\hline
2A Design & Between groups design where five groups are presented
an advertisement in the context of a webpage and asked to identify
elements with hyperlinks. Treatment groups are presented an advertisement with embedded icons at four levels (known icon different company, unknown icon, known CTA, DAA opt-out) while the control group is presented with an embedded image from the same company as the advertiser.  
\tabularnewline
2A Results & Significant difference only for the FaceBook icon. At this sample size, no difference detected between AdChoices icon and known and unknown icons.  Possibly, the sample size was too small for detection of a difference. A much larger effect was seen in the pre-pilot where advertisements were presented in isolation (not embedded in news page context).
\tabularnewline
\hline
2B Design & Between groups design where three groups are presented a digital news page and asked to click on a specific advertisement. Treatment groups are presented an advertisement
at two levels (iconic button, textual CTA) while
the control group is presented with an advertisement containing no
CTA. 
\tabularnewline
2B Results & No difference between groups. The presence of an iconic button does not affect where a user targets a click. Possibly, the effect size is too small to see.
\tabularnewline
\hline
3 Design & Between groups repeated measures design where a control
group is asked to respond to a survey containing sensitive questions. The treatment group is presented a visual indicator of ad tracker presence during presentation of each question.
\tabularnewline
3 Results & No difference between groups. It does not appear that a visual presence
indicator for advertiser activity had any affect on user behavior
with regard to propensity to disclose sensitive, personal information. It is likely that participants in both groups are equally aware of
``bystanders''. The population on AMT was not an ideal population
for this study due to very high privacy awareness and properties of
AMT, in general.
\tabularnewline
\hline
\end{longtable}
\normalsize
%\end{document}
