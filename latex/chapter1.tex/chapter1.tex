\chapter{Introduction}
\label{introduction}

 \begin{quote} 

All forms of communication entail design, as the intent of communication is to be understood by others or by one's self at another time. Communication design, then, is inherently social, because to be understood by another or by self at another time entails fashioning communications to fit the presumed mental states of others or of one's self at another time.  \citep[p. 6]{Tversky:2010ww} 
 \end{quote} 

\section{Mis-Communication in Interaction Design}
\label{mis-communicationininteractiondesign}

Good interaction design takes into account user cognitive processes and limitations. Though designers spend considerable effort toward producing user interfaces and interactions that communicate effectively, the tools that they use are flexible can easily bend to cause users to make incorrect inferences about the situation at hand. Because graphical users interfaces (GUIs) share properties with other forms of communication such as language, gesture, diagrams and action --- and are inherently interactional --- \emph{I argue that they are also subject to the same context-dependent aspects of meaning that arise with language use}. The same cognitive architecture that supports understanding of language also supports and facilitates understanding of user interfaces. This observation forms the basis for claims made in this dissertation.

In traditional GUI-based interaction, user interactions are designed. When a user interacts with a website, that interaction is structured according to that site designer's plan. Ideally, information has been organized with user expectation in mind. The designer has considered the sorts of tasks users intend to perform. Sometimes, navigation is guided. Sometimes not, but designers leave behind contextual cues and guideposts to aid users in their task.

In fact, it is easy to influence users to act in predictable ways. For example,  \cite{Tversky:1981vc}  showed that predictable, dramatic shifts in preference could be generated simply by changing the ways options are framed: in early framing experiments, subjects showed a preference toward risk aversion when lotteries were framed as gains, and risk seeking when lotteries were framed as losses. 

As noted by  \cite*{Thaler:2010uy},  designers have the potential for great control over user decision-making by manipulating behavior using psychological tools described by psychologists and behavioral economists. However, though  \cite{Kahneman:1984td}  and others have noted that small changes in context affects decision-making, there is considerably less research on how \emph{small changes in context affect understanding in the context of decision-making}.

Language operates using the same principles and processes as other cognitive functions. It is not surprising to learn that language can be used to manipulate hearer understanding. For example,
 
\begin{enumerate} 
\item[(1)] The Sheriff caught Robin red-handed. He is now serving time in prison.
\end{enumerate}

Sentence 1 above appears to convey the following facts that weren't explicitly stated:

\begin{enumerate} 
\item[(1a)] It is Robin who is serving time in prison.
\item[(1b)] Robin is serving time in prison as a result of being caught red-handed by the Sheriff.
\end{enumerate}

These un-stated propositions (1a) and (1b) are known as \emph{implicatures}, a well-known pragmatic phenomenon in language understanding. In fact, depending on situational context, either of these implicatures may not be  true.\footnote{Imagine you are watching a humorous movie and (1) is a caption. The Sheriff caught Robin Hood returning illegal taxes back to the citizenry. Later, the Sheriff was jailed for fraud. Though we might ordinarily expect the second sentence in this example to have a causal relation to the first, the visual scene serves as a contextual backdrop cancelling the implicature.}  Implicatures operate under principles such that we may presume their truth even if un-stated. But if the truth value of either implicature (1a) or (1b) turns out to be false, this does not mean that (1) is false.

Because what we know is so often derived by inference, it is easy to see how language can be used to manipulate understanding. But is it possible for GUIs to communicate in the same sort of way? I believe so. This dissertation will show that GUI-based mis-communications may be brought about by faulty inferences.

\section{Discourse Understanding}
\label{discourseunderstanding}

Discourse is generally said to encompass language use beyond the bounds of a sentence or proposition. It is often associated with the study of pragmatics: ``Syntax studies sentence. Semantics studies propositions. Pragmatics is the study of linguistic acts and the contexts in which they are performed''  \citep[p. 34]{Stalnaker:1999vl}.  Pragmatics attempt to account for meaning of messages where actual meaning is underdetermined from what is said. 

Pragmatics is fundamentally concerned with reasoning processes that go beyond conventional meaning to interpretation in social, situational, and belief contexts. As such, it is founded on the notion of language as action with communicative goals  \citep{Grice:1975vz,Levinson:2000ud,Clark:1996tm}.  In this view, ``hearers'' infer a ``speaker's'' meaning on the basis of evidence provided.

\begin{sloppier}
Discourse is also studied by sociologists, anthropologists, and sociolinguists: it is bound to socio-cultural knowledge and governed by social norms \citep{Gumperz:1982tc,Hymes:1974wr}.  Even so, discourse communication is seen as structured: both in terms of conversational events and learned, ritualistic schemas \citep[e.g., making a reservation]{Goffman:1981tm}. In this tradition, conversational inferences are also context-bound, but conceived more generally as preferences, maxims, or tendencies.
\end{sloppier}


\section{Confusion in Online Behavioral Advertising}
\label{confusioninonlinebehavioraladvertising}

Of concern to nearly everyone on the Internet today is privacy. Polls conducted over the last decade indicate that the majority of Internet users report that they do not wish to be tracked across sites  \citep{Truste:2012uc}.  At the same time, Internet advertising has entered a boom for mass data collection and predictive analytics. While it may be in the best interest of consumers to restrict online data collection and tracking, in practice, the legal system has been unable to keep pace. Even so, businesses that engage in online behavioral advertising (OBA) commonly employ techniques that influence user behavior by manipulation. The goal of advertisers, after all, is to influence consumers and sell products. 

Pertaining to OBA, the following sorts of questions are addressed in this dissertation:

\begin{itemize}
\item Why might users say privacy is important but not choose to block cookies when given a choice?

\item Why is the Interactive Advertising Bureau (IAB) AdChoices opt-out icon confusing to so many users?

\item Why might might users, who acknowledge that the Internet is not private, behave as if it is?
Such questions relate to user belief and decision-making in the context of website interactions on the Internet.

\end{itemize}

Though advertising language has been studied in the \emph{content} of ads  \citep{Leech:1966wr,Geis:1982uf,Harris:1983tj,Vestergaard:1985vn,Cook:2001up,Tanaka:1999tq},  the study of how advertisers might influence users during the course interaction has not yet been addressed. 

\section{Empirical Study}
\label{empiricalstudy}

In this dissertation, I use randomized controlled experimentation to observe user behavior under specific conditions. Each experiment asks: is there mis-understanding and is such confusion pragmatic in nature?

The first experiment hypothesizes the interpretation of \textbf{implicature} in ``do not track'' dialog boxes. Such dialog boxes are often presented as modal dialogs, thus, as an unforced ``yes-no'' choice design. In experiment 1A, I examine whether graphical representations of choice in this context evoke implicature in the same manner as textual representations. Following this, experiment 1B considers what people believe is the consequence of ``no choice'' in task-based interaction.

The second experiment focuses on the role of \textbf{deixis} in hyper-linked ad images. The Interactive Advertising Bureau (IAB) AdChoices icon associated with behavioral advertisements is designed to give users the means to control behavioral tracking preferences. However, relatively few people notice such  icons,\footnote{This is based on a national wireline and cell telephone survey of 1,503 adult Americans during April and May, 2012 \citep{TheNonTransparency:2012ut}.}  let alone click them  \citep*{TheNonTransparency:2012ut,Logic:2011wn}.   \cite*{Leon:2012dk,Ur:2012ws}  describe communicative flaws with AdChoices icon. While they considered the effect of iconic versus textual communication, they only speculate on causes for confusion. Experiment 2A hypothesizes that knowledge plays a role in whether an icon is taken to be indexically linked to a website. Experiment 2B considers whether familiar iconic representations affect how users interact with advertisements in context.

Finally, Experiment 3 focuses on the pervasive phenomenon of third-party tracking in the browser itself. In this study, user interaction on the Internet is analyzed as a form of multi-party discourse leading to to issues in \textbf{conversational inference} where the user does not infer the presence of unratified participants monitoring web interactions. Manipulating the user's perceptual awareness of third party participants is hypothesized to have an effect on user behavior. More broadly, this experiment attempts to show that user behavior, while answering sensitive questions, is affected in the presence of visible observers. I hypothesize that their propensity for disclosure of sensitive information is reduced under such conditions.

In each experiment described above, I show how advertisers may either benefit from poor design decisions or deliberately mis-lead users through the manipulation of linguistic understanding. Though the context for experimentation is situated in the domain of online behavioral advertising, the method for analysis is general in nature. 

\section{Thesis Aim}
\label{thesisaim}

The aim of this thesis is to prove that certain problems and confusions that users face interacting with technology associated with online behavioral advertising can best be understood as problems in discourse understanding. Using quantitative methods, I demonstrate that users are subject to the same context-dependent aspects of meaning that arise with the use of language. With the rise of new interactions fueled by behavioral advertising, I argue that advertisers may look toward exploiting new opportunities for manipulating meaning and understanding.

This thesis does not intend to catalog the full range of discourse phenomena extant in user interaction --- nor does it attempt to solve the problems described. It also doesn't intend to cover the entire range of phenomena within the context of OBA. The intent is to reveal how a linguistic analysis may account for user confusion not fully explained otherwise. Such confusions emphasize the need for designers to consider the importance of discourse inference in GUI-based interaction.

Chapter 2 introduces background literature on the topic of online behavioral advertising This chapter lays the foundation for analyzing observed phenomena under an inferential model of communication. Chapter 3 describes the discourse-theoretic framework necessary for testing hypotheses about pragmatic processes. Chapter 4 details a general method for study. Chapters 5, 6, and 7 each describes an experiment addressing a particular problem. Chapter 5 is concerned with whether implicature can be observed in choice processes via a non-forced choice modal dialog interaction. Chapter 6 considers knowledge and the role of deixis in hyperlinked ads. And Chapter 7 highlights the role of conversational inference in an experiment designed to ascertain whether user behavior changes in the presence of observers. Finally, Chapter 8 presents a discussion of results and offers direction for future research. It is my belief that this thesis lends deeper insight into communicative challenges for user interaction design and bridges these to sound theoretical tools for addressing a specific class of problems previously unidentified.
