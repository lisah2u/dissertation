\section{Conclusion}
\label{conclusion}

The goal of this dissertation was to prove the following:

\begin{sloppier}
\begin{enumerate}
\item Interaction with graphical users interfaces involve discourse processing affecting comprehension, and ultimately, behavior;
\item Online behavioral advertising participates in a new form of advertising discourse where user's beliefs are affected \textit{during the course of interaction}; and,
\item The use of quantitative methods lend insight into subtle problems in the design of microinteractions.
\end{enumerate}
\end{sloppier}


This dissertation has demonstrated that graphical user interface design can not only cause users to make faulty pragmatic inferences, but that interaction with such interfaces fundamentally exploits properties of the same cognitive architecture underlying linguistic competence. To my knowledge this is the first work to suggest how users of graphical user interfaces might be manipulated through small changes in context that affect understanding. If intentional, this represents what  \cite{Sperber:1986uk}  might call a form of \emph{covert communication}. Though interaction designers may be trained to recognize and understand psychological processes in user interaction, I believe equal attention is merited for the understanding of discourse processes. Both play a role in social cognition.

In addition, two other points are worthy of note.

First, it is not uncommon that technical specifications of formal standards --- such as those that drove ``Do Not Track'' and the AdChoices icon --- are developed with a focus on policy over user comprehension. It is in our best interest, as designers and users, to put such specifications under the microscope to see how they stand up in practice. Not doing so has the potential to rob these specifications of their desired effect.

Second, though advertisers have long been known as masters of manipulation of basic psychological processes, much less said about how they also manipulate meaning and understanding. Effects are arguably more subtle and difficult to see. There is rich opportunity to study so-called ``dark patterns'', in addition to more commonly used interaction design patterns, to identify potential levers and controls relating to comprehension. By doing so, we are in a better position to educate users online about how to interpret the myriad of messages and interactions to which they are exposed.
