\def\baselevelheader{0}
\maketitle
\begin{abstract}
\end{abstract}

  \begin{tabular}{ | l || c | c | }
    \hline
    Chapter & 1st Draft & 2nd Draft \\ \hline
    1. Introduction & Mar 1 & Apr 1 \\ \hline
    2. Literature Review & May 1 &  June 1 \\ \hline
    3. Theoretical Framework & Apr 1 &  May 1 \\ \hline
    4. Experiment 1 & Jun 1 & July 1 \\ \hline
    5. Experiment 2 & Jul 1 & Aug 1 \\ \hline
    6. Experiment 3 & Aug 1 & Sep 1 \\ \hline
    7. Discussion & Sep 1 & Oct 1 \\ \hline
    8. Conclusion & Oct 1 & Oct 1 \\ \hline
  \end{tabular}


\section{Introduction}
\label{introduction}

\begin{itemize}
\item Introductory paragraph - what this study will accomplish

\item Background of the problem - major findings from lit review, unresolved issues

\item Statement of problem - gap in knowledge, specific problem addressed.

\item Purpose of the study - research design (including variables, population), significance, why important

\item Research strategy - summary of approach and methods

\item Assumptions, limitations, scope

\item Thesis structure

\end{itemize}

\section{Literature Review}
\label{literaturereview}

\emph{Purpose: prove this is a gap citing major conclusions, findings, methodological issues.}

\begin{itemize}
\item Paragraph introducing major topical sections

\subsection{Privacy Issues in Online Behavioral Advertising}
\label{privacyissuesinonlinebehavioraladvertising}

\item Persistent problems, research relating to user confusion, attitude, understanding

\item Persuasive technologies

\subsection{Cognitive Features of Graphical User Interfaces}
\label{cognitivefeaturesofgraphicaluserinterfaces}

\item Review of appropriate literature in cognitive science, linguistics, and psychology

\item On the basis of reviewed literature, there is a gap in knowledge that has not yet been addressed - hypothesis that graphical user interfaces on the web evoke linguistic processes in understanding

\end{itemize}

\section{Theoretical Framework}
\label{theoreticalframework}

\emph{Purpose: review and justification for theoretical framework}

\begin{itemize}
\item Traditions in discourse analysis; two dominant theoretical and methodological approaches

\subsection{Linguistic pragmatics}
\label{linguisticpragmatics}

\subsection{Interactional discourse}
\label{interactionaldiscourse}

\subsection{Unified theory of social action}
\label{unifiedtheoryofsocialaction}

\item There exists a body of theory which accounts for problems described in the previous section. Frames experiments to follow.

\end{itemize}

\section{Experiment 1: Conversational Implicature}
\label{experiment1:conversationalimplicature}

\begin{itemize}
\item Introductory remarks

\item Structure of chapter

\end{itemize}

\subsection{Review of the Literature}
\label{reviewoftheliterature}

\emph{Purpose: Describe relevant literature on this topic}

\subsection{Aims of the Experiment}
\label{aimsoftheexperiment}

\emph{Purpose: Describe research question and basis for data collection}

\subsection{Method}
\label{method}

\emph{Purpose: Describe practices and procedures for analyzing the research question(s)}

\begin{itemize}
\item Paragraph reiterating the purpose of the study

\item Specify design while comparing with alternate methods

\item Research design - variables, level of significance used to accept \slash  reject hypothesis

\item Pilot study

\item Settings and participants

\item Instrumentation

\item Procedure

\item Data processing and analysis

\item Internal and external validity

\item Summarize the research design

\end{itemize}

\subsection{Results}
\label{results}

\emph{Purpose: summarize collected data, treatment, and analysis}

\begin{itemize}
\item Introductory paragraph

\item Results, treatment, analysis, and discussion

\item General discussion - summary of findings, insights and issues

\end{itemize}

\section{Experiment 2: Hypertext Deixis}
\label{experiment2:hypertextdeixis}

\emph{Same structure as experiment 1}

\section{Experiment 3: Conversational Inference}
\label{experiment3:conversationalinference}

\emph{Same structure as experiment 1}

\section{Discussion}
\label{discussion}

\emph{Purpose: summarize results from all experiments and discuss implications}

\begin{itemize}
\item Mention discourse phenomena not studied, yet likely to exist in practice

\end{itemize}

\section{Conclusion}
\label{conclusion}

\emph{Purpose: summary, contribution of research, recommendations for future research.}

\begin{itemize}
\item Summarize aims of thesis and research problem

\item Conclusions - summary of experimental results

\item Summary of contributions - theoretical implications

\item Future research

\end{itemize}

\section{Bibliography}
\label{bibliography}

\end{document}
