\documentclass[12pt]{article}
\usepackage{outlines}
\title{Discourse Processes in Online Behavioral Advertising}
\author{Lisa D. Harper}
\date{Version 1.0 -- February 22, 2013}

\begin{document}
\def\baselevelheader{1}
\maketitle
\begin{abstract}
\end{abstract}
\section{Schedule}
\label{schedule}


  \begin{tabular}{ | l || c | }
    \hline
    Chapter & 1st Draft \\ \hline
    Introduction & Mar 1 \\ \hline
    Literature Review & Apr 1 \\ \hline
    Methodology & May 1 \\ \hline
    Results and Discussion & Sep 1 \\ \hline
    Conclusion & Oct 1 \\ \hline
  \end{tabular}


\section{Introduction}
\label{introduction}

\emph{Purpose: highly focused review}

\begin{itemize}
\item Introductory paragraph - what this study will accomplish

\item Background of the problem (3--4 pages) - major findings from lit review, unresolved issues

\item Statement of problem (1 paragraph) - gap in knowledge, specific problem addressed.

\item Purpose of the study (3--4 paragraphs) - research design (including variables, population), significance, why important

\item Primary research question(s)(paragraph per question) - basis for data collection

\item Hypotheses (paragraph per question) - follow the research questions

\item Research design (3--4 paragraphs) - summary of methodology including participants, instrumentation, procedure

\item Assumptions, limitations, scope (1--2 paragraphs)

\item Summarize chapter 1 and outline remaining chapters

\end{itemize}

\section{Literature Review}
\label{literaturereview}

\emph{Purpose: prove this is a gap citing major conclusions, findings, methodological issues.}

\begin{itemize}
\item Paragraph introducing major topical sections

\item Privacy issues in Online Behavioral Advertising - persistent problems, research relating to user confusion, attitude, understanding

\item Cognitive features of user interfaces - review of appropriate literature in cognitive science, linguistics, and psychology

\item Discourse processes - theoretical framework

\item Conclusion stating that on the basis of reviewed literature, there is a gap in knowledge that has not yet been addressed

\end{itemize}

\section{Methodology}
\label{methodology}

\emph{Purpose: Describe practices and procedures for analyzing the research question(s)}

\begin{itemize}
\item Paragraph reiterating the purpose of the study

\item For each experiment (DNT, AdChoices, Eavesdropping)

\item Specify design while comparing with alternate methods

\item Research design - variables, level of significance used to accept \slash  reject hypothesis

\item Pilot study

\item Settings and participants

\item Instrumentation

\item Procedure

\item Data processing and analysis

\item Internal and external validity

\item Summarize the research design

\end{itemize}

\section{Results and Discussion}
\label{resultsanddiscussion}

\emph{Purpose: summarize collected data, treatment, and analysis}

\begin{itemize}
\item Introductory paragraph

\item For each experiment, results, treatment, analysis, and discussion

\item General discussion - summary of findings, insights and issues

\end{itemize}

\section{Conclusion}
\label{conclusion}

\emph{Purpose: summary, contribution of research, recommendations for future research.}

\begin{itemize}
\item Summarize aims of thesis
Conclusions - summary of experimental results

\item Summary of contributions - theoretical implications

\item Future research

\end{itemize}
\end{document}
